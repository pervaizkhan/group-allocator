\documentclass[12pt]{ORSNZ}
\usepackage{amssymb, vmargin,  ORSNZ}
\setpapersize{A4}
\setmargnohfrb{30mm}{20mm}{30mm}{20mm} % no header/footer, margins as required
\title{Partitioning of students into equitable groups: A SolverStudio solution}
\author{M. Fairley*, O. Dowson\\Department of Engineering Science\\University of
  Auckland\\New Zealand\\{*}mfai035\@@aucklanduni.ac.nz} 

\date{} % Don't include the date
\setlength{\parindent}{6mm} % 6mm paragraph indentation
\renewcommand{\baselinestretch}{1.0417} % to get 15pt lines (does it work?)

\begin{document}

\maketitle
\pagestyle{empty} \thispagestyle{empty}
\begin{abstract}
Students in their final year of a Bachelor’s in Engineering at the University of Auckland are required to complete a course known as ENGGEN403. As part of the course, students are partitioned in to groups of around 25 students and given a single week to produce a large piece of work on a given topic. In order to make this as fair and equitable as possible, the students should be partitioned in a way to make the groups are as similar as possible. This paper details the development, implementation, and results of an Excel based optimisation solution for this problem. A model was formulated to minimise a weighted combination of the greatest difference between the mean grade point average of each group, and the greatest difference between the variance of the grade point averages in each group. The model was also constrained to balance gender, ethnicity and discipline. This model was then implemented in the PuLP modelling language using the SolverStudio modelling environment. The resulting solution was given to the course organiser who used the model to partition students in the 2014 ENGGEN403 class. This paper shows how the combination of Excel and python-based optimisation languages can result in the rapid creation of an optimisation solution that requires little understanding by the end user in order to operate.

\textbf{Key words:} knapsack, pulp, solverstudio, university of auckland, group allocator
\end{abstract}

\section{The first section}

\subsection{The first subsection}

This is how to \emph{emphasize} things, do not underline!


\subsection{The second subsection}

Remember: don't use footnotes!


\section{The next section}

\subsection{Diagrams and tables}
Centre your diagrams and tables. Right-justify numbers in tables.
Tables look better if there are no vertical lines, and this is almost
always achievable. Table headings are above the table, figure headings
are below the figure.

\begin{center}

\begin{table}[h]
\caption{A table}
\vspace{0.5ex}

\begin{center}
\begin{tabular}{r r r}
\hline
0 & 1 & 1.87 \\
9 & 5.5 & 666 \\
\hline
\end{tabular}
\end{center}

\end{table}

\end{center}

\subsection{Images}

Please convert photographs and screen shots to greyscale before you convert
the file to postscript or PDF. (In Word, right click the picture, select
Format Picture, Picture, Color, Grayscale.) You will be able to preview the
image on screen as it would be printed, and the resulting PDF file will be
much smaller.

\section{Math models}
Math models should be formatted to best suit their purpose in the
paper. All parameters and variables should be clearly defined in the
paper. These definitions and explanations of the non-trival constraints
and objective functions should be included or referred to near the
model formulation.The following guidelines will help to ensure the
model is complete and well documented.

Name the model (e.g. FruitTraNZ below) for handy reference elsewhere in
the paper. Parameters, variables, and subscripts should be formatted in
math-mode (\emph{e.g.}, $x$) in running text and formatted equations.
Font style and size should be consistent with the document's text. Do
not use more than two levels of subscripting or superscripting. It is
helpful to distinguish parameters from decision variables, for instance
with parameters as upper case and decision variables as lower case.

The following example may be helpful:

\begin{description}
\item[Indices] \mbox{}\\
        $i =$ plant: $P=\{\mbox{Chch}, \mbox{Auckland}\}$;
        $j =$ wholesaler: $1,..., 5$.
\item[Parameters] \mbox{} \\
$C_{ij} = $ cost to send a box from plant $i$ to wholesaler $j$; \\
$K_i = $ production capacity of plant $i$; \\
$D_j$ = boxes of demand at wholesaler $j$.

\item[Decision variables]\mbox{} \\
$x_{ij} = $ boxes to send from plant $i$ to wholesaler $j$.

\item[Model FruitTraNZ]\mbox{} \\
Minimize \quad $\sum_{i \in P}\sum_{j=1}^5 C_{ij}x_{ij}$
\begin{equation} \label{con1}
\sum_{j=1}^5 x_{ij} =  K_i \qquad \mbox{for $i = $ Chch, Auckland.}
\end{equation}
\begin{equation} \label{con2}
\sum_{i \in P} x_{ij} \geq D_j \qquad \mbox{for $ j = 1, 2, \ldots, 5$.}
\end{equation}
\[x_{ij} \geq 0.\]

\item[Explanation] The objective is to minisime total transportation
cost. Constraint~(\ref{con1}) requires plants to produce at full
capacity. Constraint~(\ref{con2}) ensures minimum demand is satisfied
for each wholesaler.
\end{description}

\section{Readability}
Please proofread your paper and use a spelling checker. Spelling should
follow standard New Zealand, British or U.S. spelling, but must be
consistent throughout. Authors (especially new authors) are urged to
study a style manual such as the famous one of Strunk and White.

\section{Submission format}
Please submit your documents in ``printer-ready'' form, as PostScript or
Adobe PDF. Many free PDF converters are available through the web. If
you cannot produce a PostScript or PDF file, please contact the editor
of the Proceedings.

Send your file to conference@orsnz.org.nz.

\section{How to reference other articles}
Format references as in the journal Management Science. Perhaps easiest
is to use the achicago package and achicago bibstyle. An example of a
citation is \cite{dantzig-fulkerson-johnson} and \cite{kruskal}. There
are other options,  see the achicago documentation. Please avoid using
abbreviations of journals in the references.

Reference a web page in the text by its author and the date that you check
it, as University of Canterbury (2002). In the Reference section, give its
author, title, URL, and the date that you checked it. This is easily done
by manipulating the .bbl file directly if you can't work out how to do it
with BibTeX.

\section{LaTeX notes}

Included is an alteration of the standard article.cls, called ORSNZ.cls.
This ORSNZ.cls is then referred to in ORSNZ.sty which is a very slightly
altered form of achicago.sty. It is therefore safest not to change the
name of ORSNZ.cls, but others can be changed. The subsidiary files used
by achicago.sty must also be used with ORSNZ.sty. These are included in
the {\tt frankenstein} package, available from CTAN. The latter package
is included in the tarfile that includes ORSNZ.cls, etc, for convenience.
Installing {\tt frankenstein} properly is probably worth the effort,
instructions are provided. It is possible, but trickier, to avoid this.

\section{Checklist}
\begin{itemize}
\item Graphics are clean and grey-scale, with appropriately sized text.
\item References, reference format.
\item Spell-checked and proofread.
\end{itemize}


\section*{Acknowledgments}

Acknowledgments can appear in an unnumbered section preceding the
references. The Word version of this article was revised by John F.
Raffensperger, based on a version by Andrew Mason, University of
Auckland. This LaTeX version was prepared by Mark Wilson and Golbon
Zakeri, University of Auckland, and updated by Andrew Mason
and Shane Dye.


\bibliographystyle{achicago}
\bibliography{ORSNZ}
\end{document}
